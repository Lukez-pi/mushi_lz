\documentclass[11pt]{article}

\usepackage{fullpage}
\usepackage[round,semicolon]{natbib}
\usepackage{amssymb}
\usepackage{amsmath}
\usepackage{bbm}

\usepackage{hyperref}

\DeclareMathOperator{\erf}{erf}

% NOTE
% one sentence per line for nice git diffs
% WD: I'm a fan of initialed source comments like this

\title{Coalescent inference of mutation spectrum histories from nucleotide context-resolved sample frequency spectra}

\author{
W. DeWitt$^{1}$, K. Harris (if he's into it?)$^{2}$, and K. Harris$^{1}$\\
\small{Departments of $^1$Genome Sciences and $^2$Biology, University of Washington, Seattle, USA}
% \small{$^\ast$ Equal contribution}
}

\begin{document}

\maketitle

\begin{abstract}

Single nucleotide variant (SNV) frequencies partitioned according to triplet nucleotide context vary among human ancestry groups and among great ape species, indicating variation and divergence of the mutation process.
The sample frequency spectrum (SFS)---the distribution of derived allele frequencies among sampled haplotypes---is a well-studied population genetic summary statistic that is sensitive to demographic history.
We extend a coalescent framework for demographic inference from the SFS to accommodate inference of mutation spectrum histories from the triplet-resolved sample frequency spectrum (3-SFS).
We formulate inference of mutation spectrum histories from the 3-SFS as a (linear?) inverse problem.

\end{abstract}


\section*{Introduction}\label{sec:intro}

Kelley's paper \cite{Harris2017-fw} showed triplet SNV spectrum differences between human groups and between great apes, and also used a coalescent simulation approach to fit the timing of a pulse of \texttt{T\underline{C}C>T\underline{T}C} mutations in Europeans.
We want to do some coalescent theory to formulate how the expected 3-SFS depends on both demographic history (effective population size as a function of time into the past) \emph{and} triplet mutation spectrum history (triplet-specific mutation rates as a function of time into the past).
This is similar in spirit to Kelley's simulation-based approach (which assumed the European demography of \cite{?}), but is much faster, more flexibly parameterized, and extendable to joint inference of demographic and mutation spectrum histories.

The setting for our modeling is Kingman's coalescent \citep{Kingman1982-ge, Kingman1982-tf, Kingman1982-ys, Kingman2000-jr}, with all the usual niceties: neutrality, infinite sites, linkage equilibrium, and panmixia.
We will retrace the derivation by \cite{Griffiths1998-qf} of the frequency distribution of a derived allele conditioned on the demographic history, while generalizing to a time inhomogeneous mutation process.
We will make use of important results of \cite{Polanski2003-kg} and \cite{Polanski2003-ll} which facilitate numerical computation under this model, and formulate the expected 3-SFS under given demographic and mutation spectrum histories following the notation of \cite{Rosen2018-bb} for the expected SFS with a constant mutation process.


\section*{Model}\label{sec:model}

Suppose $n$ haplotypes are sampled in the present, and let rvs $T = T_2,\dots,T_n$ denote the coalescent times measured retrospectively from the present (i.e. $T_n$ is the most recent coalescent time, and $T_2$ is the TMRCA of the sample).
As in \cite{Griffiths1998-qf} \S3, we consider a marked Poisson process in which every mutation is assigned a random label drawn iid from the uniform distribution on $(0,1)$.
This is tantamount to an (uncountably) infinite sites assumption, with the unit interval representing the genome, and the random variate labels representing mutant sites.
Further suppose that mutation rate is not constant, but a specified function of time $\mu(t) > 0$ (measured in mutations per genome per generation) applying equally to all lines in the coalescent tree at a given time $t$ (measured retrospectively from the present).
A given line in the coalescent tree then acquires mutations on a genomic region $M\subset(0,1)$ at rate $|M|\mu(t)$, where $|M|$ is the Lebesgue measure of $M$ (e.g.\ if $M$ is a subinterval, $|M|$ is the length of $M$).

Let $C_h$ denote the event that there is a mutation with label $U\in(x,x+h)\subseteq(0,1)$ that subtends $b$ haplotypes in the sample, and let $I_k$ denote the event that this mutation arose at a time when the sample had $k$ ancestral lines.
For small $h$ we then have
\begin{align*}
\mathbbm{P}(C_h\mid T) &= \sum_{k=2}^n \mathbbm{P}(C_h, I_k\mid T) = \sum_{k=2}^n p_{n,k}(b)\mathbbm{P}(I_k, U\in(x,x+h)\mid T)\\
&= \sum_{k=2}^n p_{n,k}(b)\left(kh\int_{T_{k-1}}^{T_k}\mu(t)dt + O(h)\right),
\end{align*}
where $p_{n,k}(b) = \binom{n-b-1}{k-2}/\binom{n-1}{k-1}$ is the probability that a mutant that arose when there were $k$ ancestral lines of $n$ sampled haplotypes will be present in $b$ of them (see \cite{Griffiths1998-qf} eqn 1.9).
Marginalizing $T$ gives
\begin{align*}
\mathbbm{P}(C_h) &\simeq h\sum_{k=2}^n p_{n,k}(b)\ k\ \mathbbm{E}_T\left[\int_{T_{k-1}}^{T_k}\mu(t)dt\right] = h\sum_{k=2}^n p_{n,k}(b)\ k\ \left(\mathbbm{E}\left[\nu(T_{k})\right] - \mathbbm{E}\left[\nu(T_{k-1})\right]\right),
\end{align*}
where $\nu(t)\equiv\int\mu(t)dt$.
% Now, since $\mu(t) > 0$ we have $\nu(t) \nearrow$.
% With $\pi_k(t_k)$ denoting the marginal density of $T_k$, define rv $X_k \equiv \nu(T_k)$, which has density $f_{X_k}(x_k) = \pi_k(\nu^{-1}(x_k))$.
Using eqns 5-8 from \cite{Polanski2003-kg} we can write
\begin{align*}
\mathbbm{E}\left[\nu(T_{k})\right] &= \sum_{j=k}^n A_j^k\int_0^\infty\nu(t)q_j(t)dt
\end{align*}
where
\begin{align*}
A_j^k &\equiv \frac{\prod_{l=k\ne j}^{n}\binom{l}{2}}{\prod_{l=k\ne j}^{n}\left[\binom{l}{2}-\binom{j}{2}\right]}, k\le j\le n,\\
A_n^n &\equiv 1,\\
q_j(t) &\equiv \frac{\binom{j}{2}}{\eta(t)}\exp\left[-\binom{j}{2}\int_0^t\frac{dx}{\eta(x)}\right],
\end{align*}
and $\eta(t)$ is the haploid effective population size history.
Note that $q_j(t)$ is the density of the time to the first coalescent event among any subset of $j$ individuals in the present.
This can be made explicit by defining the inhomogeneous Poisson process intensity function $\lambda_j(t) = \binom{j}{2}/\eta(t)$.

The $b$-th component of the expected SFS is then

\begin{align*}
\xi_b &= \sum_{k=2}^n p_{n,k}(b) k \left(\mathbbm{E}\left[\nu(T_{k})\right] - \mathbbm{E}\left[\nu(T_{k-1})\right]\right)\\
&= \sum_{k=2}^n p_{n,k}(b) k \sum_{j=k-1}^{n}B_j^k\int_0^\infty\nu(t)q_j(t)dt
\end{align*}
where
\[
B_j^k \equiv
\begin{cases}
A_{k-1}^{k-1}, j=k-1\\
A_j^k - A_j^{k-1}, k\le j\le n\\
\end{cases}
\]
Exchanging the order of integration and summation yields a manifestly linear expression for the expected SFS as a function of the integrated mutation intensity $\nu(t)$:
\[
\xi_b = \left<\psi_b, \nu\right>,
\]
where
\begin{align*}
\psi_b(t) &\equiv \sum_{k=2}^n p_{n,k}(b)\ k \sum_{j=k-1}^{n}B_j^k q_j(t).
\end{align*}

% DEPRECATED:
%
% \subsection*{Constant MLE}\label{sec:model:constant}
%
% We initialize by fitting a constant population size.
% According to WSD's scribbles, the MLE assuming $\eta(t) = \eta_0$ (constant) is $\hat \eta_0 = \frac{S}{2 r H_{n-1}}$, where $S$ is the number of segregating sites (the sum of the observed SFS vector), $r$ is the mutation rate per genome per generation, and $H_{n-1}$ is the $n$th harmonic number.
% This was derived by using the well-known result (cited in Rosen et al.) that the expected SFS for a constant population is given by $\xi_i = \frac{2\eta_0}{i}$.
% Then the likelihood for $\eta_0$ is a Poisson random field parameterized by the $\xi_i$.
%
%
% \subsection*{The coalescent horizon: tempora incognita (see what I did there? ;))}\label{sec:model:horizon}
%
% The coalescent tree height determines the time scale within which the SFS contains demographic information.
% The expected height for n samples and a constant history $\eta(t)=\eta_0$ is the sum of the independent exponentially-distributed intercoalescent expectations:
% \[
% \mathbb{E}[T_n] = \sum_{k=2}^n \frac{\eta_0}{\binom{k}{2}} = 2\eta_0\left(1-\frac{1}{n}\right),
% \]
% where the last equality follows by recognizing the telescoping series.
% Likewise for the variance:
% \[
% \text{Var}[T_n] = \sum_{k=2}^n \left(\frac{\eta_0}{\binom{k}{2}}\right)^2 = \frac{4 \eta _0^2 \left(-6 n^2 \psi ^{(1)}(n)+\left(\left(\pi ^2-9\right) n+6\right) n+3\right)}{3 n^2},
% \]
% where $\psi ^{(1)}(n)$ is the derivative of the digamma function \url{https://en.wikipedia.org/wiki/Trigamma_function}.
%
%
% \subsection*{Asymptotically constant boundary condition}\label{sec:model:boundary}
%
% % Kelley's eqn
% In real data (or realistic simulations) the time region where the SFS has lost information is determined by the expected number of coalescent trees $\rho\ell$, where $\rho$ is the recombination rate and $\ell$ is the genome size.
% For $t$ such that the expected number of trees that have $T_{\text{MRCA}} > t$ becomes less that one, the SFS does not contain information beyond $t$.
% To wit,
% \[
% \mathbb{P}\left(T_{\text{MRCA}}>t\right) < \frac{1}{\rho \ell}.
% \]
% The P\&K model doesn't model finite genome size, so it makes sense to use the CDF of $T_{\text{MRCA}}$ in that setting.


\section*{Results}\label{sec:results}

Todo:
\begin{itemize}
\item repeat 1KG analysis like in \cite{Harris2017-fw}, see if we recapitulate Kelley's simulation-based pulse results
\item cluster triplet time series to see if we pull in minor components. Dynamic time warping?
\end{itemize}


\section*{Discussion}\label{sec:discussion}

\bibliographystyle{plainnat}
\bibliography{refs}






\end{document}
